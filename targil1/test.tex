\documentclass[]{article}
\usepackage{lmodern}
\usepackage{amssymb,amsmath}
\usepackage{ifxetex,ifluatex}
\usepackage{fixltx2e} % provides \textsubscript
\ifnum 0\ifxetex 1\fi\ifluatex 1\fi=0 % if pdftex
  \usepackage[T1]{fontenc}
  \usepackage[utf8]{inputenc}
\else % if luatex or xelatex
  \ifxetex
    \usepackage{mathspec}
  \else
    \usepackage{fontspec}
  \fi
  \defaultfontfeatures{Ligatures=TeX,Scale=MatchLowercase}
\fi
% use upquote if available, for straight quotes in verbatim environments
\IfFileExists{upquote.sty}{\usepackage{upquote}}{}
% use microtype if available
\IfFileExists{microtype.sty}{%
\usepackage{microtype}
\UseMicrotypeSet[protrusion]{basicmath} % disable protrusion for tt fonts
}{}
\usepackage[margin=1in]{geometry}
\usepackage{hyperref}
\hypersetup{unicode=true,
            pdfborder={0 0 0},
            breaklinks=true}
\urlstyle{same}  % don't use monospace font for urls
\usepackage{color}
\usepackage{fancyvrb}
\newcommand{\VerbBar}{|}
\newcommand{\VERB}{\Verb[commandchars=\\\{\}]}
\DefineVerbatimEnvironment{Highlighting}{Verbatim}{commandchars=\\\{\}}
% Add ',fontsize=\small' for more characters per line
\usepackage{framed}
\definecolor{shadecolor}{RGB}{248,248,248}
\newenvironment{Shaded}{\begin{snugshade}}{\end{snugshade}}
\newcommand{\KeywordTok}[1]{\textcolor[rgb]{0.13,0.29,0.53}{\textbf{#1}}}
\newcommand{\DataTypeTok}[1]{\textcolor[rgb]{0.13,0.29,0.53}{#1}}
\newcommand{\DecValTok}[1]{\textcolor[rgb]{0.00,0.00,0.81}{#1}}
\newcommand{\BaseNTok}[1]{\textcolor[rgb]{0.00,0.00,0.81}{#1}}
\newcommand{\FloatTok}[1]{\textcolor[rgb]{0.00,0.00,0.81}{#1}}
\newcommand{\ConstantTok}[1]{\textcolor[rgb]{0.00,0.00,0.00}{#1}}
\newcommand{\CharTok}[1]{\textcolor[rgb]{0.31,0.60,0.02}{#1}}
\newcommand{\SpecialCharTok}[1]{\textcolor[rgb]{0.00,0.00,0.00}{#1}}
\newcommand{\StringTok}[1]{\textcolor[rgb]{0.31,0.60,0.02}{#1}}
\newcommand{\VerbatimStringTok}[1]{\textcolor[rgb]{0.31,0.60,0.02}{#1}}
\newcommand{\SpecialStringTok}[1]{\textcolor[rgb]{0.31,0.60,0.02}{#1}}
\newcommand{\ImportTok}[1]{#1}
\newcommand{\CommentTok}[1]{\textcolor[rgb]{0.56,0.35,0.01}{\textit{#1}}}
\newcommand{\DocumentationTok}[1]{\textcolor[rgb]{0.56,0.35,0.01}{\textbf{\textit{#1}}}}
\newcommand{\AnnotationTok}[1]{\textcolor[rgb]{0.56,0.35,0.01}{\textbf{\textit{#1}}}}
\newcommand{\CommentVarTok}[1]{\textcolor[rgb]{0.56,0.35,0.01}{\textbf{\textit{#1}}}}
\newcommand{\OtherTok}[1]{\textcolor[rgb]{0.56,0.35,0.01}{#1}}
\newcommand{\FunctionTok}[1]{\textcolor[rgb]{0.00,0.00,0.00}{#1}}
\newcommand{\VariableTok}[1]{\textcolor[rgb]{0.00,0.00,0.00}{#1}}
\newcommand{\ControlFlowTok}[1]{\textcolor[rgb]{0.13,0.29,0.53}{\textbf{#1}}}
\newcommand{\OperatorTok}[1]{\textcolor[rgb]{0.81,0.36,0.00}{\textbf{#1}}}
\newcommand{\BuiltInTok}[1]{#1}
\newcommand{\ExtensionTok}[1]{#1}
\newcommand{\PreprocessorTok}[1]{\textcolor[rgb]{0.56,0.35,0.01}{\textit{#1}}}
\newcommand{\AttributeTok}[1]{\textcolor[rgb]{0.77,0.63,0.00}{#1}}
\newcommand{\RegionMarkerTok}[1]{#1}
\newcommand{\InformationTok}[1]{\textcolor[rgb]{0.56,0.35,0.01}{\textbf{\textit{#1}}}}
\newcommand{\WarningTok}[1]{\textcolor[rgb]{0.56,0.35,0.01}{\textbf{\textit{#1}}}}
\newcommand{\AlertTok}[1]{\textcolor[rgb]{0.94,0.16,0.16}{#1}}
\newcommand{\ErrorTok}[1]{\textcolor[rgb]{0.64,0.00,0.00}{\textbf{#1}}}
\newcommand{\NormalTok}[1]{#1}
\usepackage{graphicx,grffile}
\makeatletter
\def\maxwidth{\ifdim\Gin@nat@width>\linewidth\linewidth\else\Gin@nat@width\fi}
\def\maxheight{\ifdim\Gin@nat@height>\textheight\textheight\else\Gin@nat@height\fi}
\makeatother
% Scale images if necessary, so that they will not overflow the page
% margins by default, and it is still possible to overwrite the defaults
% using explicit options in \includegraphics[width, height, ...]{}
\setkeys{Gin}{width=\maxwidth,height=\maxheight,keepaspectratio}
\IfFileExists{parskip.sty}{%
\usepackage{parskip}
}{% else
\setlength{\parindent}{0pt}
\setlength{\parskip}{6pt plus 2pt minus 1pt}
}
\setlength{\emergencystretch}{3em}  % prevent overfull lines
\providecommand{\tightlist}{%
  \setlength{\itemsep}{0pt}\setlength{\parskip}{0pt}}
\setcounter{secnumdepth}{0}
% Redefines (sub)paragraphs to behave more like sections
\ifx\paragraph\undefined\else
\let\oldparagraph\paragraph
\renewcommand{\paragraph}[1]{\oldparagraph{#1}\mbox{}}
\fi
\ifx\subparagraph\undefined\else
\let\oldsubparagraph\subparagraph
\renewcommand{\subparagraph}[1]{\oldsubparagraph{#1}\mbox{}}
\fi

%%% Use protect on footnotes to avoid problems with footnotes in titles
\let\rmarkdownfootnote\footnote%
\def\footnote{\protect\rmarkdownfootnote}

%%% Change title format to be more compact
\usepackage{titling}

% Create subtitle command for use in maketitle
\providecommand{\subtitle}[1]{
  \posttitle{
    \begin{center}\large#1\end{center}
    }
}

\setlength{\droptitle}{-2em}

  \title{}
    \pretitle{\vspace{\droptitle}}
  \posttitle{}
    \author{}
    \preauthor{}\postauthor{}
    \date{}
    \predate{}\postdate{}
  

\begin{document}

1

\begin{Shaded}
\begin{Highlighting}[]
\NormalTok{workdir =}\StringTok{ "C:/Users/Liav/Desktop/Uni/R/targil1"}
\KeywordTok{setwd}\NormalTok{(workdir)}
\end{Highlighting}
\end{Shaded}

\subsubsection{Salaries}\label{salaries}

Now, we'll start working on the salaries file.

2.1 From a brief look on the data, we can see that the 2nd sheet on the
sal.xlsx file is just an addition of a nonimnal salary column. Therefor,
to save time and code, it will be a good idea to join the two dataframes
by adding the ``current'' column from the 2nd dataframe to the first
one. We' will use the same read\_excel command from before to do so, and
than we'll just add manually the missing column.

\begin{Shaded}
\begin{Highlighting}[]
\NormalTok{salaries =}\StringTok{ }\KeywordTok{read_excel}\NormalTok{(}\StringTok{"sal.xlsx"}\NormalTok{, }\DataTypeTok{sheet=}\DecValTok{1}\NormalTok{)}
\NormalTok{sal_nominal =}\StringTok{ }\KeywordTok{read_excel}\NormalTok{(}\StringTok{"sal.xlsx"}\NormalTok{, }\DataTypeTok{sheet=}\DecValTok{2}\NormalTok{)}
\NormalTok{salaries}\OperatorTok{$}\NormalTok{current =}\StringTok{ }\NormalTok{sal_nominal}\OperatorTok{$}\NormalTok{current}
\end{Highlighting}
\end{Shaded}

Here is the head of our new dataframe:

\begin{Shaded}
\begin{Highlighting}[]
\NormalTok{df <-}\StringTok{ }\KeywordTok{data.frame}\NormalTok{(}\DataTypeTok{x=}\DecValTok{1}\OperatorTok{:}\DecValTok{14}\NormalTok{,}\DataTypeTok{y=}\DecValTok{100}\OperatorTok{*}\NormalTok{(}\DecValTok{1}\OperatorTok{:}\DecValTok{14}\NormalTok{))}
\NormalTok{df}\OperatorTok{$}\NormalTok{y =}\StringTok{ }\KeywordTok{as.vector}\NormalTok{(}\KeywordTok{unlist}\NormalTok{(salaries[}\DecValTok{1}\NormalTok{,}\DecValTok{2}\OperatorTok{:}\DecValTok{15}\NormalTok{]))}

\NormalTok{df}\OperatorTok{$}\NormalTok{x=}\StringTok{ }\KeywordTok{as.numeric}\NormalTok{(}\KeywordTok{colnames}\NormalTok{(salaries[}\DecValTok{2}\OperatorTok{:}\DecValTok{15}\NormalTok{]))}

\NormalTok{model <-}\StringTok{ }\KeywordTok{lm}\NormalTok{(y}\OperatorTok{~}\NormalTok{x}\OperatorTok{+}\DecValTok{1}\NormalTok{,}\DataTypeTok{data=}\NormalTok{df)}
\KeywordTok{predict}\NormalTok{(model, }\DataTypeTok{newdata =}\NormalTok{ df)}
\end{Highlighting}
\end{Shaded}

\begin{verbatim}
##         1         2         3         4         5         6         7 
##  90.26450  97.89809  99.42481 100.95152 102.47824 104.00496 105.53167 
##         8         9        10        11        12        13        14 
## 107.05839 108.58511 110.11182 111.63854 113.16526 114.69197 116.21869
\end{verbatim}

\begin{Shaded}
\begin{Highlighting}[]
\CommentTok{#df$y_ip[!is.na(df$y)] <- predict.lm(mod1, data=df)}
\CommentTok{#df$y_ip2 <- predict.lm(mod1, newdata = df)}
\NormalTok{df}\OperatorTok{$}\NormalTok{y_ip <-}\StringTok{ }\KeywordTok{predict.lm}\NormalTok{(model, }\DataTypeTok{newdata =}\NormalTok{ df)}

\NormalTok{df}
\end{Highlighting}
\end{Shaded}

\begin{verbatim}
##       x        y      y_ip
## 1  2000       NA  90.26450
## 2  2005 100.0000  97.89809
## 3  2006  98.0000  99.42481
## 4  2007 100.0000 100.95152
## 5  2008 104.0000 102.47824
## 6  2009 103.0000 104.00496
## 7  2010 105.0000 105.53167
## 8  2011 107.0000 107.05839
## 9  2012 107.0000 108.58511
## 10 2013 111.0000 110.11182
## 11 2014 112.0000 111.63854
## 12 2015 114.0000 113.16526
## 13 2016       NA 114.69197
## 14 2017 116.0671 116.21869
\end{verbatim}

\begin{Shaded}
\begin{Highlighting}[]
\NormalTok{df=}
\StringTok{ }\NormalTok{df }\OperatorTok\StringTok{   }\KeywordTok{mutate}\NormalTok{(}\DataTypeTok{y =} \KeywordTok{ifelse}\NormalTok{(}\KeywordTok{is.na}\NormalTok{(y), y_ip, y))}



\NormalTok{df}
\end{Highlighting}
\end{Shaded}

\begin{verbatim}
##       x        y      y_ip
## 1  2000  90.2645  90.26450
## 2  2005 100.0000  97.89809
## 3  2006  98.0000  99.42481
## 4  2007 100.0000 100.95152
## 5  2008 104.0000 102.47824
## 6  2009 103.0000 104.00496
## 7  2010 105.0000 105.53167
## 8  2011 107.0000 107.05839
## 9  2012 107.0000 108.58511
## 10 2013 111.0000 110.11182
## 11 2014 112.0000 111.63854
## 12 2015 114.0000 113.16526
## 13 2016 114.6920 114.69197
## 14 2017 116.0671 116.21869
\end{verbatim}

\begin{Shaded}
\begin{Highlighting}[]
\CommentTok{# do stuff with row}
\end{Highlighting}
\end{Shaded}


\end{document}
